\documentclass{article}

\usepackage{amsmath}
\usepackage{hyperref}
\usepackage{graphicx}
\usepackage{booktabs}
\usepackage{amsfonts}
\usepackage{biblatex}
\usepackage{tcolorbox}
\usepackage{amsthm}
\usepackage{listings}
\usepackage{graphicx}
\usepackage{float}
\usepackage{fvextra}
\usepackage[a4paper, margin=1.2in]{geometry}


\graphicspath{ {./images/} }

\title{Reflections on Option Pricing Models}
\author{Tom Beaugé}
\date{January 27, 2025}

\tcbset{
    note/.style={
        colback=black!5,        % Background color
        colframe=black!75, % Frame color
        fonttitle=\bfseries,
        title=Note,
        arc=4mm,               % Rounded corners
        boxrule=1pt,           % Thickness of box frame
        left=6pt,
        right=6pt,
        top=6pt,
        bottom=6pt,
    }
}
\DefineVerbatimEnvironment{verbatim}{Verbatim}{breaklines=true}

\begin{document}

    \maketitle

    \begin{abstract}
        This document reflects on my self-directed exploration of financial models for option pricing, beginning with the single-period binomial tree. I discuss the theoretical foundations, implementation challenges, and insights gained from constructing and analyzing the model. This work serves as a foundation for future investigations into more complex pricing frameworks, such as multi-period trees, stochastic calculus, and market microstructure.
    \end{abstract}

    \section{Introduction}
    \label{sec:introduction}

    The pricing of financial derivatives represents a cornerstone of modern quantitative finance. My project begins with the simplest discrete-time model---the binomial tree---to build intuition about no-arbitrage pricing, risk-neutral valuation, and dynamic hedging. Though elementary, this model encodes key principles that generalize to continuous-time frameworks like the Black-Scholes-Merton model. My goal is to iteratively implement and analyze increasingly sophisticated models, using each step to deepen my understanding of stochastic processes, measure theory, and computational finance.

    \section{The Binomial Tree Model}
    \label{sec:binomial}

    With its very straightforward concept and basic mathematics, the binomial model seemed like a logical introduction to options pricing. However, this overtly simplistic binary idea that the price of an option can either go up or down is something to be noted. In the words of Paul Wilmott: "the model is for demonstration purposes only, it is not the real thing. As a model of the financial world it is too simplistic, as a concept for pricing it lacks the elegance that makes other methods preferable, and as a numerical scheme it is prehistoric. Use once and then throw away." This word of caution was kept in mind while I sought to implement it based on Chapter 15 of his book \textit{Paul Wilmott Introduces Quantitative Finance}.

    \subsection{A Simple, One-step Binomial Tree}

    In essence, the binomial model, introduced by Cox, Ross, and Rubinstein (1979), provides a discrete-time approximation of asset price dynamics, such that the underlying asset price \( S_t \) evolves in a probability space as:

    \[
        S_{t+1} =
        \begin{cases}
            S_t \cdot u & \text{with probability } p, \\
            S_t \cdot d & \text{with probability } 1 - p.
        \end{cases}
    \]

    where \( u > 1 + r > d > 0 \) ensures arbitrage-free dynamics, and \( r \) is the risk-free rate. Its simplicity belies its ability to replicate derivative payoffs through a self-financing portfolio of the underlying asset and risk-free bonds.

    \subsubsection{Implementation}

    From what was understood, the main idea of the binomial model is to create a risk-neutral valuation. This was achieved by constructing a portfolio composed of the underlying asset and risk-free bonds to exactly match the option's payoff. To prevent any arbitrage, the option's price must thus correspond to the cost of replicating this portfolio. The methodology for constructing this risk-free portfolio was directly inspired by \href{https://www.youtube.com/watch?v=eA5AtTx3rRI}{this YouTube video}, with the slight addition of fixed interest rates to account for the time value of money.

    \medskip

    The relationship governing the replicating portfolio is expressed as:

    \[
        P_t = \Delta S_t - V_t
    \]

    Here, \( P_t \) represents the value of the portfolio at time \( t \), \( \Delta \) is the hedge ratio indicating the number of units of the underlying asset held, \( S_t \) is the current price of the underlying asset, and \( V_t \) is the current price of the option. This construction ensures that the portfolio is risk-free by balancing the position in the underlying asset with the short position in the option.

    \medskip

    For the portfolio \( P_t \) to be risk-free, its future value must be identical in both the up and down states. This condition leads to:

    \[
        \Delta S_u - V_u = \Delta S_d - V_d
    \]

    Solving for the hedging ratio \( \Delta \) associated with the number of units of the underlying asset yields:

    \[
        \Delta = \frac{V_u - V_d}{S_u - S_d}
    \]

    This calculation ensures that the portfolio replicates the option's payoff in both possible future states, thereby enforcing the no-arbitrage condition.

    \subsubsection{Insights}

    By pricing the option in such a way, we are hedged against any possible fluctuation in price, as the arbitrary probability \emph{p} that the option will increase in price appears at no point in our pricing calculations.
    Playing around with the expected value, I was also able to notice that for a specific probability \emph{p}, the expected value matched the pricing of my option echoing the First Fundamental Theorem of Asset Pricing applies here.

    \begin{tcolorbox}[note, title=First Fundamental Theorem of Asset Pricing]
        The \textbf{First Fundamental Theorem of Asset Pricing} states that a financial market is \textbf{arbitrage-free} if and only if there exists at least one \textbf{equivalent martingale measure} (also known as a risk-neutral measure).

        In other words, the absence of arbitrage opportunities in a market is equivalent to the existence of a probability measure under which discounted asset prices follow a martingale process.
    \end{tcolorbox}

    \bigskip

    \subsubsection*{Deriving the risk neutral probability \emph{p}:}

    Assume the expected return equals the risk-free rate of the underlying asset:

    \[
        \mathbb{E}[S_{t+1}] = (1 + r) S_0
    \]


    where:

    \[
        \mathbb{E}[S_{t+1}] = p S_u + (1 - p) S_d
    \]

    Thus, solving for the risk-neutral probability \(p\) yields:

    \[
        p = \frac{1 + r - d}{u - d}
    \]

    \subsubsection{Limits}

    Its assumptions---constant \( u \), \( d \), \( r \), and no transaction costs---are economically restrictive.
    Moreover, the model oversimplifies price fluctuations to just an up and down movement.
    It's also completly inadapted for more complex options, very poorly estimates those with longer expirations and does not account for volatility.

    \medskip

    These limitations motivate the need for more sophisticated models.

    \subsubsection{Future Directions}

    This work lays the groundwork for several extensions:

    \begin{itemize}
        \item \textbf{Multi-Period Trees}: Generalizing to \( n \)-period trees to approximate continuous processes.
        \item \textbf{American Options}: Incorporating early exercise features via dynamic programming.
        \item \textbf{Volatility}: Experimenting with time-varying \( u \) and \( d \) to model volatility.
    \end{itemize}


    \subsection{Going Multi-step}

    The next step in my exploration was to extend the single-step binomial tree to a multi-period framework.
    This extension allows for a improvements in the modelisation of our asset price, allowing us to have a rudimentary estimate of the option's price over a larger period of time.
    The fluctuations in the asset price are still binary, but this new improvement allow us to better represent the random walk of our asset.

    \subsubsection{Implementation}

    To simplify the calculations, I have decided to only carry the theory from our previous one-step model.
    Our simple one-period option was priced using the hedge ratio \(\Delta\).
    However, to make the computation more straightforward, I have chosen to reuse the risk-neutral probability \(p\) we found earlier to price the option at each node.

    \begin{enumerate}

        \item \textbf{Building the Stock Price Tree:} A two-dimensional array is used to store the possible stock prices at maturity. At each node corresponding to \( i \) upward movements (and \( n-i \) downward movements) after \( n \) steps, the stock price is calculated as:
        \[
        S_{n,i} = S_0 \, u^i \, d^{n-i},
        \]
        where \( S_0 \) is the initial stock price.

        \item \textbf{Option Payoff Calculation:} At maturity, the option payoff is computed at each node. For a call option, the payoff is:
        \[
        \text{Payoff} = \max(S_{n,i} - K, 0),
        \]
        and for a put option:
        \[
        \text{Payoff} = \max(K - S_{n,i}, 0),
        \]
        where \( K \) is the strike price.

        \item \textbf{Backward Induction:} After setting the terminal payoffs, the tree is traversed backward to compute the option value at earlier nodes. At each node, the option value is obtained as:
        \[
        V_{i,j} = \frac{p \, V_{\text{up}} + (1 - p) \, V_{\text{down}}}{1 + r},
        \]
        where:
        \begin{itemize}
            \item \( V_{\text{up}} \) is the option value from the node corresponding to an upward movement,
            \item \( V_{\text{down}} \) is the option value from the node corresponding to a downward movement,
            \item \( 1 + r \) is used to incorporate the risk-free rate per period.
        \end{itemize}
        This backward induction continues until the root node is reached, where the option price at time zero is obtained.
    \end{enumerate}


    A short code excerpt is shown below:

    \small
    \begin{verbatim}
    public MultiStepBinomialTree(double initialPrice, double strikePrice,
            double probabilityUp, double upFactor, double downFactor,
            double interestRate, boolean isCall, int steps) {

        // Calculate risk-neutral probability: q = (1 + r - d) / (u - d)
        double q = calculateRiskNeutralProbability(upFactor, downFactor, interestRate);

        // Initialize arrays for stock prices and option values
        stockPriceMaturity = new double[steps + 1][steps + 1];
        optionValues = new double[steps + 1][steps + 1];

        // Backward induction to build the tree and calculate option values
        for (int step = steps; step >= 0; step--) {
            for (int i = 0; i <= step; i++) {
                // Calculate stock price at the node: S = initialPrice * u^i * d^(step-i)
                stockPriceMaturity[step][i] = initialPrice * Math.pow(upFactor, i) * Math.pow(downFactor, step - i);

                // Set option payoff at maturity or compute the option value by backward induction
                if (step == steps) {
                    optionValues[step][i] = calculateOptionPayoff(stockPriceMaturity[step][i], strikePrice, isCall);
                } else {
                    optionValues[step][i] = calculateOptionValue(optionValues[step + 1][i], optionValues[step + 1][i + 1], interestRate, q);
                }
            }
        }
    }
    \end{verbatim}

    \normalsize

    \emph{Note:} The code snippet is a simplified version of the actual implementation, which includes additional error handling and input validation.

    \medbreak

    The final option price is available as the value of the root node of the tree, i.e., \texttt{optionValues[0][0]}. This represents the present value of the option under the given parameters and assumptions of the binomial model.

    \medbreak

    \textbf{Example Case}
    To illustrate the implementation, consider the following example:

    \begin{itemize}
        \item Initial stock price (\( S_0 \)): \$100
        \item Strike price (\( K \)): \$105
        \item Up factor (\( u \)): 1.1
        \item Down factor (\( d \)): 0.9
        \item Risk-free rate (\( r \)): 5\% per period
        \item Number of steps: 3
        \item Option type: Put
    \end{itemize}

    \clearpage

    \begin{figure}[H]
        \centering
        \includegraphics[scale=0.4]{Example1_BinTree_OptionValue}
        \caption{Option values for a 3-step tree}
        \label{fig:3_step_option_tree}
    \end{figure}

    \begin{figure}[H]
        \centering
        \includegraphics[scale=0.4]{Example1_BinTree_StockPrice}
        \caption{Stock prices}
        \label{fig:3_step_stock_price_tree}
    \end{figure}

    The rightmost nodes in Figure~\ref{fig:3_step_option_tree} show the payoffs of the put option at expiration, computed as:
    \[
    P_T = \max(K - S_T, 0)
    \]
    Some nodes have a value of \$0.00, indicating that the stock price was above the strike price at expiration, making the put option worthless.

    \medskip

    The deeper the stock price falls below \( K = 105 \), the higher the put option value. The highest option price is observed in paths where the stock price significantly declines.

    The root node of the tree gives an option price of \$17.49, which represents the fair value of the put option under these assumptions.

    \subsubsection{Insights}


    The following figure illustrates the evolution of computation time as the number of steps increases in the multi-step binomial tree model.

    \begin{figure}[h]
        \centering
        \includegraphics[width=0.8\textwidth]{ComputationTime_BinTree_v1}
        \caption{Computation time of each step in the binomial tree model (1000 steps)}
        \label{fig:computation_time_bintree_v1}
    \end{figure}

    The observed quadratic trend in computation time suggests that the model’s complexity scales poorly with large steps. Optimization strategies such as dynamic programming, parallel processing, and memory-efficient algorithms should be explored to improve performance.
    If we look at the code snippet above, the time complexity is dominated by the nested loops:

    \bigbreak

    Outer loop: Iterates over steps from \( n \) to \( 0 \) (total \( n + 1 \) steps).
    Inner loop: For each step \( s \), iterates \( i \) from \( 0 \) to \( s \) (total \( s + 1 \) iterations per step).

    \medbreak
    Per-node operations:

    For each node \( ( \text{step}, i ) \), \texttt{calculateOptionValue} computes the option value using:

    \[
        \text{Value} = \frac{q \cdot \text{Payoff}_{\text{Up}} + (1 - q) \cdot \text{Payoff}_{\text{Down}}}{1 + r}
    \]

    Nonetheless, while most operations to calculate the payoff is constant time (\emp{see code above}), \emph{Math.pow} is of complexity \( O(\log n) \).

    \bigskip

    This simplifies to a total time complexity of:
    \[
    O(n^2 \log n)
    \]

    As a general benchmark, the current implementation computes around 1100 steps in 10s with the memory usage that maxes out around 225MB.
    The current challenge is to enhance computational speed while simultaneously optimizing memory efficiency.

    \subsubsection{Speeding up the process}

    Given the quadratic growth in computation time and the substantial memory footprint, several optimizations were implemented to enhance both speed and memory efficiency in the multi-step binomial tree model.

    \paragraph{1. Precomputing Powers}
    Repeated calls to \texttt{Math.pow} were identified as a significant performance bottleneck, given their logarithmic time complexity \(O(\log n)\).
    To address this the increase and decrease factors are now precomputed using induction:

    \begin{verbatim}
        double[] upPowers = new double[steps + 1];
        double[] downPowers = new double[steps + 1];
        upPowers[0] = 1.0;
        downPowers[0] = 1.0;
        for (int j = 1; j <= steps; j++) {
            upPowers[j] = upPowers[j - 1] * upFactor;
            downPowers[j] = downPowers[j - 1] * downFactor;
        }
    \end{verbatim}

    This optimization reduces the total time complexity from:

    \[
        O(n^2 \log n) \quad \text{to} \quad O(n^2)
    \]

    This results in a major time-related improvement - 2800 steps are now computed in 10s - but now significantly more memory is used, peaking around 1600MB.

    \paragraph{2. Allocating jagged arrays}
    Instead of assigning a square array for our stock price and option value through time, leading to unnecessary empty entries, each row was allocated with the exact size they needed (i.e.\ step number + 1)
    \midskip
    This led to some improvements in the memory usage which now maxes around 1300MB. Its time efficienty also increased reaching 3100 in 10 seconds.
    This is hypothesized to be due to a reduction in garbage collection pressure as well as a better cache utilisation since no superfluous data is present.

    \paragraph{2. Separating Terminal and Backward Induction Computations}
    The original implementation used conditional checks inside nested loops to differentiate terminal payoff calculations from backward induction. These conditionals introduce unnecessary branching overhead.

    \begin{itemize}
        \item **Terminal Nodes:** Handled separately before backward induction begins.
        \item **Backward Induction:** Simplified to eliminate conditional checks, improving CPU branch prediction.
    \end{itemize}

    \paragraph{3. Memory Optimization}
    Originally, two-dimensional arrays were used to store option values and stock prices, leading to significant memory overhead.

    \begin{itemize}
        \item **One-dimensional Arrays:** Since backward induction only requires values from the current and next step, a one-dimensional array suffices, overwriting values in place.
        \item **Triangular Memory Structure:** Alternatively, jagged arrays can be used, allocating memory only where needed.
    \end{itemize}

    This reduces memory usage from \(O(n^2)\) to \(O(n)\) for option values.

    \paragraph{4. Potential for Parallelization}
    The terminal node computations are independent and thus highly parallelizable.

    \begin{itemize}
        \item **Multi-threading:** Utilized for terminal payoff calculations.
        \item **Task-based Parallelism:** Possible in backward induction with careful dependency management.
    \end{itemize}

    \paragraph{5. Benchmarking Optimized Implementation}

    \begin{itemize}
        \item **Computation Time:** Reduced from 10s for 1100 steps to approximately 3.2s.
        \item **Memory Usage:** Reduced from 225MB to around 120MB.
    \end{itemize}

    \paragraph{Summary}
    These optimizations improve the model's scalability, making it more efficient for large-scale financial computations without sacrificing accuracy.



\end{document}
